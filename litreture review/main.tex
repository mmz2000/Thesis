\documentclass{beamer}

\title{Social Intelligence in Computer Vision: A Review of Cues, Models, and Applications}
\author{Mohmmad M. Zare'i}
\institute{University of Tehran}
\date{\today}

\begin{document}
\frame{\titlepage}
\begin{frame}
    \frametitle{Outline}
    \tableofcontents
\end{frame}


\section{Introduction}
\begin{frame}
    \frametitle{Introduction}
    \begin{itemize}
        \item \textbf{Activity Recognition (Individual)} Identifying what a person is doing.
              Datasets: UCF101, Kinetics


        \item \textbf{Group Activity Recognition (Social)} Understanding coordinated or interacting actions within groups.
              Datasets: Collective Activity, Volleyball Dataset


        \item \textbf{Why It Matters}
              \begin{itemize}
                  \item Moves beyond visual labels $\rightarrow$ requires reasoning about \textit{roles, relationships, intentions}.
                  \item Foundation for social intelligence in computer vision.
                  \item Applications: autonomous driving, social robotics, surveillance, human–AI collaboration.
              \end{itemize}
    \end{itemize}
\end{frame}

\subsection{Complexity of Group Activities}
\begin{frame}{Complexity Beyond Actions}
    \begin{itemize}
        \item Group activities involve \textbf{multiple people} and their interactions.
        \item Requires modeling of:
              \begin{itemize}
                  \item \textit{Roles} (leader, follower, bystander)
                  \item \textit{Relationships} (friend, rival, teammate)
                  \item \textit{Intentions} (cooperation, competition, avoidance)
                  \item \textit{Context} (sports field, meeting room, street)
              \end{itemize}
        \item Moves beyond simple action recognition $\rightarrow$ toward \textbf{social reasoning}.
    \end{itemize}
\end{frame}

\subsection{Social and Cognitive Cues}
\begin{frame}{Social and Cognitive Cues}
    \begin{itemize}
        \item Key cues from \textbf{social psychology} and \textbf{cognitive science}:
              \begin{itemize}
                  \item Facial expressions $\rightarrow$ emotions
                  \item Body language $\rightarrow$ posture, gestures
                  \item Gaze direction $\rightarrow$ attention, focus
                  \item Proxemics $\rightarrow$ distance and spatial relationships
                  \item Turn-taking $\rightarrow$ conversational dynamics
              \end{itemize}
        \item These cues provide the foundation for \textbf{interpreting group interactions}.
    \end{itemize}
\end{frame}

\subsection{Technical Challenges}
\begin{frame}{Technical Challenges}
    \begin{itemize}
        \item Multi-person detection and tracking across frames.
        \item Temporal reasoning: modeling interactions over time.
        \item Multimodal integration: vision + audio + language.
        \item Ambiguity and context dependence of group behaviors.
        \item Generalization: handling unseen group dynamics.
    \end{itemize}
\end{frame}

\subsection{From Recognition to Understanding}
\begin{frame}{From Recognition to Understanding}
    \begin{itemize}
        \item Traditional goal: \textbf{recognize what is happening}.
        \item Emerging goal: \textbf{understand why it is happening}.
              \begin{itemize}
                  \item Infer group goals, intentions, and social context.
                  \item Connect low-level cues with high-level reasoning.
              \end{itemize}
        \item This leap is the essence of \textbf{social intelligence in video}.
    \end{itemize}
\end{frame}

\section{Literature Review}
\begin{frame}
    \frametitle{Literature Review}
    \begin{itemize}
        \item Overview of key works in social intelligence and computer vision.
        \item Discussion of methodologies, datasets, and findings.
        \item Identification of gaps and future directions.
    \end{itemize}
\end{frame}

\subsection{SoGAR: Self-supervised Spatiotemporal Attention-based Social Group Activity Recognition}
\begin{frame}
    \frametitle{SoGAR: Self-supervised Spatiotemporal Attention-based Social Group Activity Recognition}
    \begin{itemize}
        \item \textbf{Datasets}: Volleyball Dataset, JRDB-PAR, NBA Dataset
        \item \textbf{Key Contributions}:
              \begin{itemize}
                  \item Introduced a self-supervised learning framework.
                  \item Leveraged spatiotemporal attention for improved interaction modeling.
                  \item Demonstrated effectiveness on multiple group activity datasets.
              \end{itemize}
        \item \textbf{Model Architecture}:
              \begin{itemize}
                  \item Base: Vision Transformer (ViT) and TimeSformer
                  \item Learning: Self-supervised pretraining on large video datasets
              \end{itemize}
              % Vision Transformer + TimeSformer
    \end{itemize}
\end{frame}

\begin{frame}{SoGAR}
    \begin{itemize}
        \item \textbf{Methodology}:
        \item Uses a self-supervised transformer framework (Vision Transformer backbone) for social group activity recognition.
        \item Key idea: generate \textbf{local} and \textbf{global} spatio-temporal views from the same video, with variation in frame rate and spatial crop size.
        \item A teacher-student architecture: teacher processes global view, student processes local views. The student is trained to align its features to those of the teacher.
        \item Two contrastive / correspondence objectives:
              \begin{itemize}
                  \item Temporal Collaborative Learning (TCL): relate views differing in temporal resolution.
                  \item Spatio-temporal Cooperative Learning (SCL): relate views that differ in spatial crop + temporal sampling.
              \end{itemize}
        \item Does \textbf{not} require actor bounding boxes or individual action labels during pre-training, reducing annotation burden.
        \item Uses motion as supervisory signal from RGB alone; the model becomes invariant to scale, viewpoint, and motion speed.

    \end{itemize}
\end{frame}

\subsection{LaIAR: Language-Model-Guided Interpretable Video Action Reasoning}
\begin{frame}
    \frametitle{LaIAR: Language-Model-Guided Interpretable Video Action Reasoning}
    \begin{itemize}
        \item \textbf{Datasets}: Charades, CAD-120
        \item \textbf{Key Contributions}:
              \begin{itemize}
                  \item Proposed a framework that integrates large language models (LLMs) for interpretable action reasoning. Utilizing knowledge transfer between LLMs and video model.
                  \item Utilized LLMs to generate explanations and rationales for recognized actions. Inferring high-level actions from low-level changes in relationships between actors and objects.
              \end{itemize}
        \item \textbf{Model Architecture}:
              \begin{itemize}
                  \item R-CNN and ResNet-101 as backbones for object, category, and relation detection.
                  \item Relations and visual features are mapped to a joint embedding space.
                  \item Embeddings are fed into a dynamic token transformer (DT-Former).
              \end{itemize}
    \end{itemize}
\end{frame}

\begin{frame}{LaIAR}
    \begin{itemize}
        \item \textbf{Methodology}:
        \item Introduces a dual-branch framework: a video model and a language model, trained together so the video model learns reasoning from the language model.
        \item Uses relationship transitions between humans/objects as cues: visual relations (appearance, bounding boxes, spatial configuration) and semantic relations (human/object categories + relationships) are encoded.
        \item Both visual and semantic relations are encoded via Faster R-CNN + ResNet-101 for detecting entities, extracting features, forming human-object pairs.
        \item Core architecture: DT-Former (Dynamic Token Transformer) which applies adaptive token selection (spatio-temporal tokens) and then transformer layers to model relation transitions. Tokens with low importance are discarded via a Gumbel-Softmax mechanism.
        \item Learning scheme includes:
              \begin{itemize}
                  \item Joint visual-semantic embedding: aligning visual relation features and semantic relation features into a common space.
                  \item Token selection supervision: encourage video branch to select similar important relations as the language branch.
                  \item Cross-modal learning: video model tries to mimic both classification outputs and relational reasoning patterns of the language model.
              \end{itemize}
        \item Interpretability built-in: at inference time only video model is used, but it produces evidence (which relations it focused on; transitions between relationships) that can be mapped to semantic labels.
    \end{itemize}
\end{frame}
\end{document}